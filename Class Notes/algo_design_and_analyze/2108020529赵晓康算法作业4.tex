\documentclass{article}
\usepackage[left=2cm, right=2cm, top=0.8in, bottom=0.7in]{geometry}

% \usepackage[utf8]{inputenc} % 用于指定文件的编码格式  
% \usepackage[fontset=ubuntu]{ctex} % 使用ctex宏包并指定字体集

\usepackage{xeCJK}
\setmainfont{MiSans}
\setCJKmainfont{MiSans}
\setCJKfamilyfont{song}{MiSans}
\setCJKmonofont{MiSans}
\usepackage{setspace}
\usepackage[linesnumbered,ruled,vlined]{algorithm2e}
\onehalfspacing

\usepackage{color}
\usepackage{graphicx} % 导入图像的宏包、单图
\usepackage{subfigure} % 导入图像的宏包、子图
\usepackage{listings}
\lstset{breaklines}%这条命令可以让LaTeX自动将长的代码行换行排版
\lstset{extendedchars=false}%这一条命令可以解决代码跨页时,章节标题,页眉等汉字不显示的问题
\lstset{
        language=C++, %用于设置语言为C++
	keywordstyle=\color{blue} \bfseries,
	identifierstyle=,
	basicstyle=\small\ttfamily, 
	commentstyle=\color{green}\textit,
	stringstyle=\color{red}\ttfamily, 
	showstringspaces=false,
        tabsize=2,
	%frame=shadowbox, %边框
	numbers=left,
        numberstyle=\small\color{magenta},
}

\begin{document}

\title{算法作业4}
\author{2108020529 赵晓康} % 替换成自己的学号和姓名
\maketitle

% 每道题包括题目问题描述和解答。答案请填下解答区。
\section{算法设计题}
\subsection{问题描述}

给定数组$A[1..n]$和一个下标$k(1\leq k \leq n-1)$,现在需要将数组的左右两部分以下标$k$为分界线调换位置。例如,输入数组$A[1..7]=\{1,2,3,4,5,6,7\}$,下标为$k=4$,我们的目标是将数组变为$A^{\prime}[1..7]=\{5,6,7,1,2,3,4\}$。

\begin{enumerate}
    \item 请设计一个时间复杂度为$O(n^2)$,空间复杂度为$O(1)$的算法来解决这个问题;
    \item 请设计一个时间复杂度为$O(n)$,空间复杂度为$O(n)$的算法来解决这个问题;
    \item 请设计一个时间复杂度为$O(n)$,空间复杂度为$O(1)$的算法来解决这个问题;
\end{enumerate}

\subsection{解答}
\subsubsection{时间复杂度为$O(n^2)$,空间复杂度为$O(1)$的算法}
\begin{lstlisting}
    #include <iostream>
#include <vector>
#include <algorithm>

// 时间复杂度为 O(n^2),空间复杂度为 O(1) 的算法
void swapSegmentsQuadratic(std::vector<int>& A, int k) {
    int n = A.size();
    for (int i = 0; i < k; ++i) {
        int temp = A[0];
        for (int j = 0; j < n - 1; ++j) {
            A[j] = A[j + 1];
        }
        A[n - 1] = temp;
    }
}

int main() {
    std::vector<int> A = {1, 2, 3, 4, 5, 6, 7};
    int k = 4;

    swapSegmentsQuadratic(A, k);

    for (int i : A) {
        std::cout << i << " ";
    }
    std::cout << std::endl;

    return 0;
}
\end{lstlisting}
\subsubsection{时间复杂度为$O(n)$,空间复杂度为$O(n)$的算法}
\begin{lstlisting}
    #include <iostream>
#include <vector>
#include <algorithm>

// 时间复杂度为 O(n),空间复杂度为 O(n) 的算法
void swapSegmentsLinearExtraSpace(std::vector<int>& A, int k) {
    int n = A.size();
    std::vector<int> B(n);

    for (int i = 0; i < n - k; ++i) {
        B[i] = A[k + i];
    }
    for (int i = 0; i < k; ++i) {
        B[n - k + i] = A[i];
    }

    A = B;
}

int main() {
    std::vector<int> A = {1, 2, 3, 4, 5, 6, 7};
    int k = 4;

    swapSegmentsLinearExtraSpace(A, k);

    for (int i : A) {
        std::cout << i << " ";
    }
    std::cout << std::endl;

    return 0;
}
\end{lstlisting}
\subsubsection{时间复杂度为$O(n)$,空间复杂度为$O(1)$的算法}
\begin{lstlisting}
    #include <iostream>
#include <vector>
#include <algorithm>

// 反转数组的一部分
void reverse(std::vector<int>& A, int start, int end) {
    while (start < end) {
        std::swap(A[start], A[end]);
        ++start;
        --end;
    }
}

// 时间复杂度为 O(n),空间复杂度为 O(1) 的算法
void swapSegmentsLinearInPlace(std::vector<int>& A, int k) {
    int n = A.size();
    reverse(A, 0, n - 1);
    reverse(A, 0, n - k - 1);
    reverse(A, n - k, n - 1);
}

int main() {
    std::vector<int> A = {1, 2, 3, 4, 5, 6, 7};
    int k = 4;

    swapSegmentsLinearInPlace(A, k);

    for (int i : A) {
        std::cout << i << " ";
    }
    std::cout << std::endl;

    return 0;
}
\end{lstlisting}

%--------------------------------
\section{微博名人问题}
\subsection{问题描述}

给定$n$个人。我们称一个人是“微博名人”,如果他被其他所有人微博关注,但是自己不关注任何人。为了从给定的$n$个人中找出名人,唯一可以进行的操作是:针对两个人$A$和$B$,询问“A是否微博关注B”。答案智能是YES(A关注B)或NO(A不关注B)。

\begin{enumerate}
    \item 在一群共$n$个人中,可能有多少个名人?
    \item 请设计一个算法找出名人(你可以很容易实现一个基于遍历的算法,请尝试在此基础上进行改进)。
\end{enumerate}

\subsection{解答}
\subsubsection{可能有多少个名人}
在一群共 $n$ 个人中,最多只能有一个名人。

根据题目的定义,一个人是“微博名人”的条件是:
他被其他所有人微博关注。
他自己不关注任何人。

如果存在两个或更多的名人,假设有两个人 $A$ 和 $B$ 都是名人,那么根据定义:
$A$ 必须被所有人关注,包括 $B$。
$B$ 也必须被所有人关注,包括 $A$。
这就产生了矛盾,因为名人自己不关注任何人。因此,在一群人中,最多只能有一个名人
\subsubsection{找出名人}
\begin{lstlisting}
    int findCelebrity(int n) {
    int candidate = 0;

    // Step 1: Find the candidate
    for (int i = 1; i < n; ++i) {
        if (knows(candidate, i)) {
            candidate = i;
        }
    }

    // Step 2: Verify the candidate
    for (int i = 0; i < n; ++i) {
        if (i != candidate && (knows(candidate, i) || !knows(i, candidate))) {
            return -1; // No celebrity
        }
    }

    return candidate; // Celebrity found
}
\end{lstlisting}
逐步排除:通过询问每对候选人和其他人的关系,逐步排除不可能是名人的人。这个过程最多进行 $n-1$ 次比较。
验证候选人:验证最终的候选人是否满足名人的条件。这个过程需要进行 $2(n-1)$ 次比较
%--------------------------------
\section{参数传递代价}

\subsection{问题描述}

我们在分析算法时有一个假设——过程调用中的参数传递花费常量时间,即使传递一个$N$个元素的数组也是如此。在大多数系统中,这个假设是成立的,因为传递的是指向数组的指针,而非数组本身。但也有其他可能的参数传递策略:

\begin{enumerate}
    \item 数组通过指针传递,时间为$\Theta(1)$;
    \item 数组通过元素复制来传递,时间为$\Theta(N)$,$N$是数组的规模;
    \item 传递数组时,只复制过程可能访问的子区域。例如,子数组$A[p..q]$被传递,则时间为$\Theta(q-p+1)$.
\end{enumerate}

\begin{itemize}
    \item 考虑在有序数组中超找元素的递归二分查找算法。请分别给出在上述三种数组参数传递策略下,二分查找的最坏情况运行时间递归表达式,并给出递归式解的上界。令$N$为原问题的规模,$n$为子问题的规模。
    \item 类似地,讨论归并排序算法的表现情况。
\end{itemize}

\subsection{解答}
%--------------------------------
\section{快速排序的再研究}

\subsection{问题描述}

对快速排序算法,考虑如下的\textit{pivot}选取方案:在所有待排序的元素中,均匀随机选取3个元素,将这3个元素的中位数定位\textit{pivot}。

\begin{enumerate}
    \item 请计算这一方法所选中的\textit{pivot}正好是所有元素中第$i$小的元素的概率;
    \item 这一新方法中输入元素的中位数为\textit{pivot}的概率,是原始QUICK-SORT中选中中位数为\textit{pivot}概率的多少倍?这一比率的极限是多少(令$n\rightarrow \infty$)?
    \item 这一方法是否能在渐进增长率上改进经典的QUICK-SORT算法?
\end{enumerate}


\subsection{解答}
%--------------------------------
\section{温故}

\subsection{问题描述}

给定正整数数组$A=[92,23,8,46,26,43,6,97]$,请分别使用插入排序、冒泡排序、快速排序与归并排序,对该数组进行排序。请用过程图展示各个算法的详细过程(图解方法可参考算法导论)。

\subsection{解答}

%--------------------------------
\section{然后知新}

\subsection{问题描述}

给定一个由整数组成的序列$S$,请找出和最大的连续子序列。例如,$S=\{-2,11,-4,13,-5,-2\}$,得到的结果应为$20=11-4+13$。

\begin{enumerate}
    \item 你可以基于简单遍历数组元素,设计一个$O(n^3)$的算法。
    \item 改进上述算法中的冗余计算,可以得到一个$O(n^2)$的基于遍历的算法。
    \item 基于分治策略,你可以设计一个$O(n\log n)$的算法。
    \item 分析遍历算法中的冗余计算,可以设计一个$O(n)$的算法。
    \item (可选)基于动态规划策略,同样可以得到一个$O(n)$的动态规划算法。
\end{enumerate}

\subsection{解答}

\end{document}